\section{Introduction}

In order to push the limits of visual fidelity, modern games are shaded with
ever increasing number of lights and shaders with more complexity. These shading
calculations make up a significant percentage of the amount of work to render a
single frame. To facilitate these complex shading calculations in a real-time
application, it is important that the number of light calculations is kept to
a minimum. There are several different approaches to reducing this workload.
Light assignment in one of such approaches. It is a technique that reduces the
number of light calculations per pixel be excluding lights that do not influence
a pixel. This is possible due to representing lights as having a finite volume
of influence. Ideally, per pixel only lights which are contained in the light
volume are evaluated for that pixel.

In modern games two light assignment algorithms are used commonly. Tiled
Shading\cite{} and Clustered Shading\cite{}. Both rely on a subdivision
of the view frustum to assign lights to pixels. This requires a complete
recalculation of the associated data structures per frame. With a high frame
rate of between 30 and 60 frames a second, most changes between frames are small
and local. Thus a significant part of the light assignment data structures could
potentially be reused.

We introduce the Hashed Shading algorithm, which does not require a complete
rebuild of the associated data structures per frame, in order to explore
whether reusing data structures per frame is a feasible approach that leads
to a performance gain compared to Tiled and Clustered Shading.

We make use of a camera-independent octree data structure to subdivide the scene
space. This camera-independent data structure allows us to reuse the data structures
between frames, even if the camera position changes.

