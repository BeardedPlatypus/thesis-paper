\section{Introduction}

In order to push the limits of visual fidelity in modern games, the complexity
of shaders and the number of lights keeps increasing. These complex calculations
require significant processing time. To facilitate frames being rendered in
real-time, the amount of unnecessary light calculations need to be kept to a
minimum. There are several approaches to reduce this amount of work, while
preserving the accuracy of the shading. One of such approaches is Light
Assignment. This technique reduces the number of light calculations per pixel by
excluding lights that do not influence it. This is possible due to the finite
representation of lights generally used within real-time applications. Ideally
per pixel only lights of which the corresponding light volume contains the pixel
are evaluated. This reduces the number of light calculations to the bare minimum.

In recent games two light assignment algorithms are used commonly, Tiled
Shading\cite{olsson2011tiled} and Clustered Shading\cite{olsson2012clustered}.
Both algorithms subdivide the view frustum. For each sub frustum the set of
lights which overlap with the volume of the sub frustum are calculated and
subsequently retrieved during shading to the reduce the number of lights which
need to be evaluated.

This direct dependency on the view frustum requires the data structures of the
two light assignment algorithms to be recalculated per frame. Due to the high
frame rate at which images in real-time applications are generated, changes in
the scene are generally small and local. Significant parts of the generated data
structures could potentially be reused in order to reduce computation time and
improve performance.

The goal of this paper is to explore whether reusing the generated data
structures does indeed yield a performant algorithm. For this purpose the Hashed
Shading algorithm is introduced. This algorithm makes use of a view frustum
independent subdivision of the scene space, such that a change in camera
position or orientation does not warrant a complete rebuild of the data
structures. The subdivision itself is build upon the octree data structure, such
that the scene space is represented both precise and memory efficient.

In order to validate whether data structures persistent between frames can
potentially achieve better performance than current light assignment data
structures the Hashed Shading algorithm needs to achieve at least a similar
execution time speed up and light calculation reduction as Tiled and Clustered
Shading.



% In order to push the limits of visual fidelity, modern games are shaded with
% ever increasing number of lights and shaders with more complexity. These shading
% calculations make up a significant percentage of the amount of work to render a
% single frame. To facilitate these complex shading calculations in a real-time
% application, it is important that the number of light calculations is kept to
% a minimum. There are several different approaches to reducing this workload.
% Light assignment in one of such approaches. It is a technique that reduces the
% number of light calculations per pixel be excluding lights that do not influence
% a pixel. This is possible due to representing lights as having a finite volume
% of influence. Ideally, per pixel only lights which are contained in the light
% volume are evaluated for that pixel.

% In modern games two light assignment algorithms are used commonly. Tiled
% Shading\cite{} and Clustered Shading\cite{}. Both rely on a subdivision
% of the view frustum to assign lights to pixels. This requires a complete
% recalculation of the associated data structures per frame. With a high frame
% rate of between 30 and 60 frames a second, most changes between frames are small
% and local. Thus a significant part of the light assignment data structures could
% potentially be reused.

% We introduce the Hashed Shading algorithm, which does not require a complete
% rebuild of the associated data structures per frame, in order to explore
% whether reusing data structures per frame is a feasible approach that leads
% to a performance gain compared to Tiled and Clustered Shading.

% We make use of a camera-independent octree data structure to subdivide the scene
% space. This camera-independent data structure allows us to reuse the data structures
% between frames, even if the camera position changes.

