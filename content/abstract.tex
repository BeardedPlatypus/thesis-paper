\begin{abstract}

  This paper introduces the Hashed Shading algorithm, a light assignment algorithm
  for forward and deferred shading. It uses a subdivision independent from the view
  frustum in order to reuse data structures between frames. The scene is subdivided
  into cubes of a specified size and represented using a linkless octree to store
  the data efficiently in memory and allow for a fast retrieval of relevant lights
  during shading.
  The performance of Hashed Shading is compared to both Tiled and Clustered Shading.
  Forward Hashed Shading reduces the number of light calculations and the execution
  time by a factor of two compared to Forward Tiled Shading. It achieves a similar reduction
  in the number of light calculations as Clustered Shading. Furthermore the Hashed
  Shading algorithm scales slightly better in regards to resolution than Tiled and
  Clustered Shading due to the camera-independent subdivision.
  Currently Hashed Shading does not support dynamic lights, and requires significant
  memory to store the linkless octree. Solutions for these issues are proposed but
  have not been implemented.

\end{abstract}
