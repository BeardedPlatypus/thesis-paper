\section{Related Work}

\subsection{Deferred Shading}

A first step into reducing the number of light calculations is decoupling
geometry and shading complexity. In the modern graphical pipeline, all geometry
in the view-frustum generates samples, regardless of whether it is occluded.
These samples are shaded and only after shading the visible samples are
determined in the per-pixel operations. In order to just shade the visible
samples, the visibility of the samples has to be determined before shading.
This concept was introduced in hardware design in 1988\cite{} and is called
Deferred Shading. In 1990 a general purpose strategy was introduced based on so
called GBuffers\cite{}. The idea is to render all attributes needed to shade a
sample into texture buffers the size of the screen. Then in a second, deferred
step these values are retrieved from the buffers to calculate the shading
contribution.

A first light assignment approach is to rasterise the light volumes, and
only calculate the contribution of the light for the generated samples.
This technique is called the stencil-optimalisation\cite{}. This reduces
the number of light calculations, as no longer for each sample each light
has to be evaluated, however there are two also two disadvantages of this
optimalisation. Firstly it requires a significant number of memory
accesses, as for each generated sample, the GBuffers have to be sampled.
Secondly the light contributions are accumulated in a buffer, thus leading
to small rounding errors.

\subsection{Tiled Shading}

Tiled Shading was designed to alleviate the memory bandwidth bottleneck of
Deferred Shading with the stencil-optimalisation. The concept is to divide
the screen space into tiles of $\mathit{n} \times \mathit{n}$ pixels. For
each tile the light indices of the overlapping light volumes are stored.
The tiles are subsequently used to determine the set of relevant lights per
sample by retrieving the set of light indices associated with the tile the
sample falls within. This technique thus only requires a single memory access
per sample, instead of a memory access per light, thus reducing the memory
bandwidth significantly.

The tiles can be constructed by projecting the light volumes on the view port
and determining with which tiles it overlaps. The corresponding light index
is added to each of these tiles. A further optimalisation is to reduce the
tile volume by only considering lights that fall within the minimum and
maximum $z$-value of the tile. Any light that does not overlap with this
sub frustum, is not added to the tile.

This technique has been implemented in a number of games\cite{} and game engines\cite{}, including
the unreal engine\cite{}, the frostbyte engine\cite{} and unity\cite{}.

\subsection{Clustered Shading}

Clustered Shading\cite{} extends Tiled Shading by subdividing the space associated
with each tile further. The tiles are subdivided exponentially in the camera
$z$-axis to gain cube-like sub frustums. For each of the sub frustums containing
geometry, the overlapping lights are calculated. Furthermore additional
attributes like normals can be used to create even more precise light assignment.

The subdivision is constructed by first determining the set of clusters which
contain geometry. This is done by transforming the depth buffer in a cluster
buffer, by assigning each sample to a distinct bin. Then the unique clusters
are obtained by locally sorting and compacting each tile. The cluster buffer
is then used to create a mapping from each sample to the unique cluster to
which it belongs. For each of the unique clusters the overlapping lights
are added to construct a set of light indices.

During the shading step the set of light indices is retrieved using the
sample to cluster mapping constructed during the construction of the data
structures. For each of these lights the contributions is calculated and
summed similarly to Tiled Shading.

Clustered Shading is used in several modern games, including in the Avalance
game engine, on which games as Just Cause 2 and 3\cite{} and Doom 4\cite{} are build.

\subsection{Perfect Spatial Hashing}

In order to efficiently subdivide the scene space without using a large amount
of memory an octree data structure is used. A straightforward implementation
using pointers would require a large number of control structures. Control structures
can lead to a significant performance loss on the GPU\cite{}. Thus such using such
an octree would not be feasible in a light assignment algorithm. A different
approach is needed. The linkless octree\cite{} data structure used within this paper
is based upon perfect spatial hash functions\cite{}.

Hash functions can be used to map sparse data $D$ on to a compact memory table $H$.

\begin{equation*}
  D\left(\mathbf{p}\right) = H\left[ \mathit{h}\left(\mathbf{p} \right) \right]
\end{equation*}

\noindent When a hash function is perfect no keys map to the same address. 
This feature can be used on the GPU as it will ensure no control structures are needed to extract a
value associated with a key from memory.

The goal of perfect spatial hashing to hash multi-dimensional sparse data onto a compact
memory table, without having any collisions and maintaining spatial coherence.
This makes the data structure fitted for applications on the graphics card.
The perfect spatial hash function $\mathit{h}$ is build with two imperfect hash functions,
$\mathit{h}_0$ and $\mathit{h}_1$ and an offset hash table $\Phi$. The idea is to dissolve
collisions in $\mathit{h}_0$ with the offset values in $\Phi$. This leads to the following
definition:

\begin{equation*}
  \mathit{h}\left(\mathbf{p}\right) = \mathit{h}_0\left(\mathbf{p}\right) + \Phi\left[\mathit{h}_1\left(\mathbf{p}\right)\right]
\end{equation*}

\noindent where $\mathit{h}_0$ and $\mathit{h}_1$ are simple hash functions defined as:

\begin{align*}
  \mathit{h}_0\left(\mathbf{p}\right) =& \mathbf{p} \operatorname{mod} \dot{m} \\
  \mathit{h}_1\left(\mathbf{p}\right) =& \mathbf{p} \operatorname{mod} \dot{r}
\end{align*}

\noindent where $\dot{m}$ is the dimension of hash table $H$ and $\dot{r}$ the dimension
of hash table $\Phi$.

This representation makes it possible to represent sparse three dimensional data efficiently
in memory. The linkless octree data structure will be build with these perfect spatial hash
functions.

\subsection{Linkless Octree}

The linkless octree is a GPU efficient octree implementation, which is utilised in the
Hashed Shading algorithm. It represents each layer of the octree as a perfect spatial
hash function.

An octree contains a number of layers, each layer has a defined node size, thus the position
of each node can be represented by a three dimensional integer vector. There exists a node
at position $\mathbf{p}_l$ in layer $\mathit{l}$ if and only if at layer $\mathit{l} - 1$
there is a branch node containing the node at position $\mathbf{p}_l$ or $\mathbf{p}_l$ is the
root node of the octree. This means that each layer consists of a set of nodes sparsely distributed
over the octree space. Each layer can thus be efficiently represented by a perfect spatial hash function.

In order to encode a layer, the data associated with a node needs to be defined. This data encodes
the type of an octree node at position $\mathbf{p}$. An octree node can either be a branch
node or a leaf node, and an octree node can be either empty or non-empty. Thus each octree node
can be represented by two boolean values. Due to memory considerations eight nodes are combined
and represented by two 8-bit integers. Since each branch node contains eight children, each
linkless octree node in layer $\mathit{l}$ represents a branch node containing the types of
its eight children at depth $\mathit{l}+1$.
A complete octree can be represented by constructing a perfect spatial hash function for each
layer.

The data saved in the octree node needs to be stored as well. There are two options to do this.
Either the data is stored directly in the hash table of the perfect spatial hash function, or
the data is stored separately. In case the data is stored directly inside the hash table,
each hash table entry requires the memory to hold two 8-bit integers and eight times the
memory of a single data element associated with an octree node. This approach is infeasible
when the data is large, or only a small subset of nodes actually contains data. In these
cases the data can be saved in a separate perfect spatial hash function. The hash table
associated with this second perfect spatial hash function will contain an entry for each non-empty
child node of the nodes saved in the first spatial hash function.

