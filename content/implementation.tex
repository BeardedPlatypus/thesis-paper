\section{Implementation}

In order to evaluate the Hashed Shading algorithm a new program, \texttt{nTiled},
was developed. This program was build with \texttt{C++} and \texttt{openGL} and
compiled with Visual Studio 2015. 
Besides the Hashed Shading algorithm also a naive implementation which evaluates each
light per sample, and the Tiled and Clustered Shading algorithms were implemented.
All programs execute the same shading code, a simple white lambertian material.

\subsection{Deferred Shading}

Deferred is implemented by executing two render passes. In the first pass
the normal, a white albedo colour, and the depth are written to a GBuffer containing
the corresponding textures. In a second pass a full screen quad is rendered, and for
each sample the shading information is retrieved from the GBuffer and the
light contributions calculated.

\subsection{Tiled Shading}

The forward and deferred Tiled Shading shaders make use of the ... algorithm to
map the light volumes on the screen space. This is done on the CPU. The implementation
does not take into account the minimum and maximum $z$-values within tiles, thus
the tile volumes are not optimised with regards to these values.

\subsection{Clustered Shading}

Clustered Shading was implemented without the bounding volume hierarchy specified in
the original algorithm, the lights instead were assigned in a brute force fashion by
adjusting the Tiled Shading algorithm, and evaluating whether it overlapped with any
clusters which contained geometry. The sort and compact step were implemented with
\texttt{openGL} compute shaders. The assignment of lights was implemented on the CPU
which requires the use of \texttt{glGetImage}\footnote{}. This leads to a significant
slow down of the algorithm which skews the execution time results.

\section{Evaluation}
% Add information about how the nodde sizes are related
% Latter compared to Tiled and Clustered Shading

% Construction Time and Memory Usage
The performance of Hashed Shading algorithm is evaluated with regards to both its construction
and the execution. For the construction the construction time is evaluated. Furthermore the
number of pixels used in the perfect spatial hash functions and the size of the light index
list are evaluated.

% Light calculations and execution time
The execution time of both the Forward and the Deferred Pipeline are compared to the Tiled and
naive implementations. Due to complications with the Clustered Shading implementation, its
execution time is not taken into account. The number of light calculations per frame for
the deferred naive implementation, deferred tiled shading, deferred clustered shading and
deferred hashed shading were also compared.

% 3 scenes used
All tests were done for three scenes. An indoor spaceship scene was based on CG Lighting
Challenge \#18\footnote{}. A street scene was based on CG Lighting Challenge \#42\footnote{}.
And a city scene based on a scene from the open movie Sintel\footnote{}. These scenes were
chosen to represent different possible game environments. For each scene light generation
volumes were defined. Within these volumes different number of lights were generated, to
create different number of lights for each scene.

% each value evaluated with regards to the resolution and lights averaged over whole executions
The influence of both the number of lights, as the resolution on the execution time and
number of lights of the different light assignment algorithms is evaluated. For this six
different resolutions ranging from $320 \times 320$ to $2560 and 2560$ are evaluated, and
number of lights ranging from less than a hundred till more than a thousand lights per
scene are evaluated.

All of these simulations were executed on a Dell XPS laptop with the following hardware:

\begin{table}[h!]
  \begin{tabular}{@{}ll@{}}\toprule
    Operating system       & Windows 10 64-bit                \\
    CPU & Inter Core i7 6700 HQ @ 2.60 Ghz \\
    Memory & 16 GB                            \\
    GPU & NVIDIA GeForce GTX 960M          \\
    GPU drivers & 372.70                           \\ \bottomrule
  \end{tabular}
\end{table}









% Hardware used

% Analysis done with seaborne and pandas

